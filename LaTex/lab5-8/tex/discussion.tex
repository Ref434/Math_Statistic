\documentclass[../main.tex]{subfiles}
\begin{document}
    \subsection{Выборочные коэффициенты корреляции и эллипсы рассеивания}
    \begin{itemize}
        \item Для двумерного нормального распределения дисперсии выборочных коэффициентов корреляции упорядочены следующим образом: $r < r_{S} < r_{Q}$; для смеси распределений получили обратную картину: $r_{Q} < r_{S} < r$.
        \item Процент попавших элементов выборки в эллипс рассеивания (95$\%$-ная доверительная область) примерно равен его теоретическому значению (95$\%$).
    \end{itemize}
    \subsection{Оценки коэффициентов линейной регрессии}
    \begin{itemize}
        \item Критерий наименьших квадратов точнее оценивает коэффициенты линейной регрессии на выборке без возмущений.
        \item Критерий наименьших модулей точнее оценивает коэффициенты линейной регрессии на выборке с возмущениями.
        \item Критерий наименьших модулей устойчив к редким выбросам.
    \end{itemize}

    \subsection{Проверка гипотезы о законе распределения генеральной совокупности. Метод хи-квадрат}
        \noindent 
        Для всех случаев справедливо неравенство $\chi^2_B<\chi^2_{0.95}$. Для случая выборки, распределенной по закону $N(x,\hat{\mu},\hat{\sigma})$, можно сказать, что гипотеза $H_0$ о нормальном распределении на уровне значимости $\alpha=0.05$ согласуется с ней с определенной точностью. Для случаев распределения Лапласа и равномерного распределения эта оценка немного хуже приближается к параметрам нормального распределения, что объясняется малым количеством выборки. В итоге для всех случаев гипотеза $H_0$ принята.

    \subsection{Доверительные интервалы для параметров распределения}
    \begin{itemize}
        \item Генеральные характеристики ($m$ = 0 и $\sigma$ = 1) накрываются построенными доверительными интервалами. 
        \item Доверительные интервалы, полученные по большей выборке, являются соответственно более точными, т.е. меньшими по длине. 
        \item Доверительные интервалы для параметров нормального распределения более надёжны, так как основаны на точном, а не асимптотическом распределении.
\end{itemize}
\end{document}